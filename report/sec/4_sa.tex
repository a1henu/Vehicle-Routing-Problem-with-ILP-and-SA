\section{模拟退火方法求解VRP问题}

\subsection{搜索问题简介}

搜索问题指的是在一个问题空间中寻找解决方案的过程。这个问题空间可以是各种形式的,例如图形、状态集合或搜索树等。搜索问题通常由以下要素构成:

\begin{enumerate}
    \item 初始状态(Initial State): 描述问题的起始状态。
    \item 目标状态(Goal State): 描述问题的目标或解决方案。
    \item 操作(Actions): 定义在每个状态下可以采取的行动或操作。
    \item 状态转移(State Transition): 描述从一个状态到另一个状态的转移规则,即在执行某个操作后,如何从一个状态转移到下一个状态。
    \item 路径成本(Path Cost): 定义从一个状态到另一个状态的成本。
\end{enumerate}

对于一般的搜索问题,我们可以先形式化地描述问题,然后使用搜索算法来寻找问题的解决方案。

一般地,搜索算法的目标是找到从初始状态到目标状态的路径,使得路径的成本最小。
我们通常会从一个初始状态开始,然后通过不断地尝试各种操作,直到找到目标状态为止。

\begin{algorithm}[H]
    \SetKwInOut{Input}{Input}
    \SetKwInOut{Output}{Output}
    
    \Input{Initial State}
    \Output{A path to the goal state}
    
    \While{not reach the goal state}{
        Choose an action\;
        Execute the action\;
        Update the state\;
    }

    \Return{Path to the goal state}

    \caption{General Search Algorithm}
\end{algorithm}

\subsection{模拟退火算法}

模拟退火(Simulated Annealing,SA)是一种启发式搜索算法,用于在大规模搜索空间中寻找全局最优解。
它的灵感来源于固体物理中的退火过程,即通过逐渐降低温度来使得物质达到低能量状态。
在搜索问题中,"能量"可以被理解为目标函数的值,"温度"控制了搜索过程中接受更差解的概率。

在这个算法中,我们从一个初始状态开始,然后在每一步中生成当前状态的一个邻居。
如果这个邻居的成本更低,或者即使成本更高但根据当前的温度和成本差异计算出的概率\ref{sec_2:sa_possibility}
仍然决定接受这个邻居,我们就将当前状态更新为这个邻居。
然后我们降低温度,并重复这个过程,直到温度降到0为止。最后,我们返回当前状态作为找到的最优解。

我们可以将模拟退火算法的过程简述如下:

\begin{algorithm}[H]
    \SetKwInOut{Input}{Input}
    \SetKwInOut{Output}{Output}
    
    \Input{Initial State, Initial Temperature, Cooling Rate, Cutoff Temperature}
    \Output{Optimal State}
    
    current\_state $\leftarrow$ Initial State\;
    temperature $\leftarrow$ Initial Temperature\;
    \While{temperature > Cutoff Temperature}{
        next\_state $\leftarrow$ Generate a neighbor of current\_state\;
        cost\_difference $\leftarrow$ Cost of next\_state - Cost of current\_state\;
        \If{cost\_difference < 0 or exp(-cost\_difference / temperature) > random number between 0 and 1}{
            current\_state $\leftarrow$ next\_state\;
        }
        temperature $\leftarrow$ temperature * Cooling Rate\;
    }

    \Return{current\_state}

    \caption{Simulated Annealing Algorithm}
    \label{sa_algo}
\end{algorithm}

\subsection{VRP问题的搜索问题建模}

对于VRP问题这样的NP-hard问题,我们可以将其建模为一个搜索问题,使用启发式搜索算法来寻找解决方案。

在VRP问题中,我们可以将每个状态定义为一个车辆的路径,每个操作定义为对路径的修改。
我们可以通过不断地调整车辆的路径,直到找到一个最优的解决方案。

根据\ref{sa_algo},我们需要定义以下要素:

\begin{enumerate}
    \item 初始状态: 每个车辆的路径都是从配送中心出发,经过若干客户点,最终回到配送中心。
    \item 目标状态: 每个车辆的路径都满足约束条件,使得总成本最小。
    \item 操作: 对车辆的路径进行调整,例如交换两个客户点的顺序。
    \item 状态转移: 调整车辆的路径。
    \item 路径成本: 每个车辆的路径的总成本。
\end{enumerate}

在算法实现中,我们需要确定初始状态的生成方式,确定当前状态的邻居的生成方式,
确定如何计算当前状态的成本,以及确定如何根据当前状态的成本和温度来决定是否接受邻居。

对于初始状态,我们一般通过随机排列的方式来生成;对于邻居的生成,我们可以通过交换或插入客户点来实现。
为了行文方便,我们采用下面记号:

\textit{getNext}: 生成当前状态的邻居。

\textit{getCost}: 计算当前状态的成本。

\textit{accept}: 根据当前状态的成本和温度来决定是否接受邻居。

\subsection{模拟退火算法求解VRP问题}

有了上文的框架,应用模拟退火求解VRP就变得十分简单了,我们可以直接套用模拟退火算法的框架\ref{sa_algo},只需要实现上述的三个函数即可。

需要注意的是,由于我们的邻居是通过随机交换或插入客户点来生成的,因此我们不一定在每一轮迭代中都能找到更优的解,
为了保证算法的收敛性,我们在每一轮退火时都多次搜索迭代邻居,这样可以增加解收敛的概率。

\begin{algorithm}[H]
    \SetKwInOut{Input}{Input}
    \SetKwInOut{Output}{Output}
    
    \Input{Initial State, Initial Temperature, Cooling Rate, Cutoff Temperature}
    \Output{Optimal State}

    \SetKwFunction{FAc}{accept}
    \SetKwProg{Fn}{Function}{:}{end}

    current\_state $\leftarrow$ Initial State\;
    current\_cost $\leftarrow$ getCost(current\_state)\;
    temperature $\leftarrow$ Initial Temperature\;
    \While{temperature > Cutoff Temperature}{
        \For{$i \leftarrow 1$ \KwTo $N$}{
            next\_state $\leftarrow$ getNext(current\_state)\;
            next\_cost $\leftarrow$ getCost(next\_state)\;
            cost\_diff $\leftarrow$ next\_cost - current\_cost\;
            \If{accept(cost\_diff, temperature)}{
                current\_state $\leftarrow$ next\_state\;
                current\_cost $\leftarrow$ next\_cost\;
            }
        }
        temperature $\leftarrow$ temperature * Cooling Rate\;
    }
    \Return{current\_state}

    \Fn{\FAc{cost\_diff, temperature}}{
        \If{cost\_diff < 0}{
            \Return{True}
        }
        \Return{exp(-cost\_diff / temperature) > random number between 0 and 1}
    }

    \caption{Simulated Annealing Algorithm for VRP}
\end{algorithm}

\subsection{模拟退火方法求解CVRP问题}

当我们考虑CVRP问题时,我们需要增加约束条件,即车辆的容量约束。

上文我们提到,我们需要应用\textit{getNext}函数来生成当前状态的邻居,
但我们在生成邻居时采用了随机的方式,如何保证生成的邻居满足约束条件呢?

我们可以在每次随机时应用\textit{success}函数来判断当前状态是否满足约束条件,如果不满足,则重新生成。

\begin{algorithm}[H]
    \SetKwInOut{Input}{Input}
    \SetKwInOut{Output}{Output}

    \Input{Current State}
    \Output{Next State}

    \SetKwFunction{FNext}{getNext}
    \SetKwProg{Fn}{Function}{:}{end}

    \Fn{\FNext(current\_state)}{
        next\_state $\leftarrow$ copy of current\_state\;
        \While{True}{
            Randomly select two customer points\;
            current\_state $\leftarrow$ Swap the two customer points or insert one point to another\;
            \If{success(current\_state)}{
                break\;
            }
        }
        \Return{current\_state}
    }

    \caption{Generate Neighbor for VRP}
\end{algorithm}

那么,无论我们给VRP问题添加什么约束条件,只要我们能够修改\textit{success}函数与其他相关数据结构的定义,
我们就可以应用一般搜索问题的框架,进而使用模拟退火算法来求解。