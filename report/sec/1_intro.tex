\section{问题介绍}
现在,我们有一定数量的客户,各自有一定的货物需求,我们有一个配送中心用于分发货物。
配送中心用一个车队分送货物,我们需要对各车辆分配适当的行车路线,使得客户的需求得到满足,
并在一定的约束下,达到诸如路程尽可能短、成本尽可能小、耗费时间尽可能少等目的。这就是车辆路径问题(Vehicle Routing Problem,VRP)。

车辆路径问题是一类常见的组合优化问题,同时也是NP-hard问题。在本研究中,我们使用北太天元与MATLAB软件,辅以python辅助编程尝试解决如下两个问题:

\begin{itemize}
    \item 基本车辆路径问题(VRP):
    设某配送中心有$m$辆车,需要对$n$个客户进行配送,客户$i$与客户$j$的距离为$c_{ij}$。
    每辆车都需要从配送中心出发给若干客户送货,最终回到配送中心。求车辆行驶路程最短的配送方案。
    \item 容量限制车辆路径问题(Capacitated Vehicle Routing Problem, CVRP):
    设某配送中心有$m$辆车,每辆配送车的最大载重量为$Q_i$,配送中心需要对$n$个客户进行配送,
    客户点$i$的货物重量为$d_i$,客户$i$与客户$j$的距离为$c_{ij}$。每辆车都需要从配送中心出发给若干客户送货,
    最终回到配送中心。求车辆行驶路程最短的配送方案。
\end{itemize}

特别地,旅行商问题(Traveling Salesman Problem, TSP)\footnote{
    旅行商问题,即假设有一个旅行商人需要经过一系列城市,
    他需要规划适当的路径,从某个城市出发,恰好每个城市经过一次,
    并最终回到出发的城市,使得整个旅行过程中的总距离最小。
    旅行商问题也是NP-hard问题。
}
也属于车辆路径问题,TSP问题是VRP问题在$m=1$情况下的特例。

根据具体的应用场景和约束条件,除了带基本车辆路径问题和带容量限制的车辆路径问题外,VRP还有一系列常见的变体,现列举部分如下:

\begin{itemize}
    \item 带时间窗的车辆路径问题(Vehicle Routing Problem with Time Windows, VRPTW):
    每个客户都有一个可接受的服务时间范围,车辆需要在规定的时间内到达客户处进行服务。
    \item 多车型车辆路径问题(Heterogeneous Fleet Vehicle Routing Problem, HFVRP):
    使用多种不同类型的车辆进行配送,每种车辆可能有不同的容量和成本。
    \item 动态车辆路径问题(Dynamic Vehicle Routing Problem, DVRP):
    在配送过程中,客户需求或交通状况可能会发生变化,需要实时调整车辆路径。
\end{itemize}

车辆路径问题被广泛应用于物流、配送、运输等领域,它的研究不仅具有理论价值,还有重要的实际意义。
车辆路线的实际问题包括配送中心配送、公共汽车路线制定、信件和报纸投递、航空和铁路时间表安排、工业废品收集等。 

随着经济全球化和信息化进程的不断加快,物流作为具有广阔前景和增值功能的新兴服务业,正在全球范围内迅速发展。
运输服务是物流组成中的重要环节,降低运输成本、提高运输质量和效率成为加快物流发展的有效途径。
车辆路径问题作为运输服务优化的核心问题,对于提高物流系统的效率和效益具有重要意义。
