\section{算法简介}
VRP问题有很多求解算法,包括精确算法与启发式算法等。
精确算法有分支定界法、动态规划法\cite{kara2003integer}等;
启发式解法有遗传算法\cite{baker2003genetic}、模拟退火\cite{vincent2017simulated}、
蚁群算法、与基于神经网络的机器学习解法\cite{bi2022learning}等。
由于VRP是NP-hard问题,我们难以用精确算法求解,因此启发式算法是求解车辆运输问题的主要方法。

本研究中,我们采用精确算法中的0-1整数规划和启发式解法中的模拟退火算法进行求解。以下是对两个算法的简要介绍。

\begin{itemize}
    \item 0-1整数规划

    规划类问题是常见的数学建模问题,常见的规划模型包括线性规划、非线性规划和整数规划三个种。
    当规划中的变量限制为整数时,称为整数规划。0-1整数规划是整数规划的一个子问题,这里所有的决策变量$x_i$\footnote{
        这里讨论一般情形,后文具体问题中我们使用$x_{ij}^{k}$表示决策变量。
    }
    只能取$0$或$1$这两个值,这里决策变量又可称为二进制变量。0-1整数规划及一些常见的规划模型为许多科学与工程问题的解决提供了基础性的支撑。
    \item 模拟退火算法

    模拟退火算法(Simulated Annealing, SA)是一种基于概率的随机寻优算法,模拟了物理中固体物质的退火过程:
    在加温阶段,固体内部粒子随温度上升变得无序,内能增大;在冷却阶段,粒子逐渐趋于有序,内能减小,最终达到常温时的基态,此时内能最小。

    模拟退火算法从某一较高初始温度出发,伴随温度参数的不断下降,结合一定的概率突跳特性,随机寻找目标函数的全局最优解,
    即在局部最优中能有概率地跳出并最终趋于全局最优。

    模拟退火算法的过程简述如下:
    
    \begin{enumerate}
        \item 初始化各种参数,如初始温度,最终温度,退火速率等。
        \item 重复以下步骤:
        \begin{enumerate}
            \item 产生新路径(新解);
            \item 计算新的总距离$x^{\prime}$与当前路径的总距离$x$;
            \item 以概率$P$决定是否接受新路径;
        \end{enumerate}
        \item 逐步降低温度T。
    \end{enumerate}
    直至温度降至截断温度,算法结束。

    上面算法中的概率$P$由Metropolis准则给出:
    \begin{equation}
        P(x^{\prime}, x) = \begin{cases}
            1, & \text{if } x^{\prime} < x \\
            \exp\left(-\frac{x^{\prime} - x}{T}\right), & \text{otherwise}
        \end{cases}
        \label{sec_2:sa_possibility}
    \end{equation}
    
    这里$x$是当前时刻的系统能量(目标函数),$x^{\prime}$是新解下的系统能量(目标函数)。
    这意味着在搜索过程中,算法会根据一定的概率接受较差的解,从而避免陷入局部最优。

    模拟退火算法是一种通用的优化算法,具有概率的全局优化性能。
    目前已在工程中得到了广泛应用,如TSP问题、VRP问题、VLSI布局问题、排课问题、机器学习中的参数优化、图像处理等。
\end{itemize}
