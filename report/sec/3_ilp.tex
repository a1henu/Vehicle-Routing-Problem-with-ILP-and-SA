\section{整数线性规划法求解VRP问题}

\subsection{问题建模}

\subsubsection{基本车辆路径问题}
    给定一组客户点、车辆容量、车辆数量、起始点和终点,目标是找到使得所有客户点都被访问一次的最短路径方案。
    基本车辆路径问题(Vehicle Routing Problem, VRP)的数学模型可以使用整数线性规划(Integer Linear Programming, ILP)来表示。
    以下是一个简化的VRP模型示例:\\
    \textbf{\textit{参数}}:\\
    $n$: 客户点的数量(不包括起始点)\\
    $m$: 车辆的数量\\
    $c_{ij}$: 客户点$i$到客户点$j$的距离或成本\\
    \textbf{\textit{变量}}:\\
    $x_{ij}^{k}$: 车辆$k$在访问客户点$i$后移动到客户点$j$,则变量$x_{i,j}^k$是1,否则是0.\\
    \textbf{\textit{目标函数}}:
    \[\min\sum_{k=1}^{m}\sum_{i=0}^{n}\sum_{j=0}^{n}c_{ij}\cdot x_{ij}^{k}\]
    \textbf{\textit{约束条件}}:\\
    1. 每个客户点必须被访问且仅被访问一次:
    \[\sum_{k=1}^{m}\sum_{j=1}^{n}x_{ij}^{k} = 1, \forall i = 1, \dots, n\]
    2. 每辆车的路径必须从起始点开始,并在终点结束:
    \[\sum_{j = 1}^{n}x_{0j}^{k} = 1, \forall k = 1, \dots, m\]
    \[\sum_{i = 1}^{n}x_{i0}^{k} = 1, \forall k = 1, \dots, m\]
    3. Binary(0-1)约束:
    \[x_{ij}^{k}\in \{0, 1\}, \forall i = 0, \dots, n, \forall j = 0, \dots, n, \forall k = 1, \dots, m\]
    4. 每辆车不能停留在同一个客户点:
    \[x_{ii}^k = 0, \forall i = 1, \dots, n+1, \forall k = 1, \dots, m\]
\subsubsection{容量限制车辆路径问题(CVRP)}
与基本VRP类似,但每个客户点有一个特定的需求量,车辆需要满足总容量限制且在不超过容量的情况下为客户提供服务。
    \\
    \textbf{\textit{参数}}:\\
    $n$: 客户点的数量(不包括起始点)\\
    $m$: 车辆的数量\\
    $Q$: 每辆车的容量限制\\
    $d_i$: 客户点$i$的需求量\\
    $c_{ij}$: 客户点$i$到客户点$j$的距离或成本\\
    \\
    \textbf{\textit{变量}}:\\
    $x_{ij}^{k}$: 车辆$k$在访问客户点$i$后移动到客户点$j$,则变量$x_{i,j}^k$是1,否则是0.\\
    \\
    \textbf{\textit{目标函数}}:
    \[\min\sum_{k=1}^{m}\sum_{i=0}^{n}\sum_{j=0}^{n}c_{ij}\cdot x_{ij}^{k}\]
    \textbf{\textit{约束条件}}:\\
    1. 每个客户点必须被访问且仅被访问一次:
    \[\sum_{k=1}^{m}\sum_{j=1}^{n}x_{ij}^{k} = 1, \forall i = 1, \dots, n\]
    2. 每辆车的路径必须从起始点开始,并在终点结束:
    \[\sum_{j = 1}^{n}x_{0j}^{k} = 1, \forall k = 1, \dots, m\]
    \[\sum_{i = 1}^{n}x_{i0}^{k} = 1, \forall k = 1, \dots, m\]
    3. 车辆容量限制:
    \[\sum_{j = 1}^{n}d_{j}\cdot x_{ij}^{k} \leq Q, \forall i = 1, \dots, n, \forall k = 1, \dots, m\]
    4. Binary(0-1)约束:
    \[x_{ij}^{k}\in \{0, 1\}, \forall i = 0, \dots, n, \forall j = 0, \dots, n, \forall k = 1, \dots, m\]
    5. 每辆车不能停留在同一个客户点:
    \[x_{ii}^k = 0, \forall i = 1, \dots, n+1, \forall k = 1, \dots, m\]

\subsection{算法}
    因为CVRP问题是VRP问题的一个延伸,因此我们在讨论解决ILP问题的算法时可以只讨论对于CVRP问题的算法。
\subsubsection{数据载入}
    我们编写了一个简单的python程序来随机生成测试数据, 以便于我们调试代码. 在代码中,我们将所有的城市(客户点)分在了四个区域内,这将有利于后续进行的车辆路线的可视化。
    代码的输出将会把数据存储在一个相同目录下的$.mat$,我们只需要将数据在$.m$文件中直接将数据导入。
\subsubsection{数据处理}
除了现有的$n$个城市,我们还需要考虑货车的出发点,所以需要处理邻接矩阵。由于下面的演示示例中车辆起点到每一个客户点的距离已经定义,否则对于n阶方阵c来说,需要添加矩阵元素$c_{0j}$和$c_{i0}$;由于车辆的总数是m,所以要产生m个分量,每一个分量都是一个二维的矩阵$c$。
    \begin{lstlisting}
    x = optimvar('x', n+1, n+1, m, 'Type', 'integer', 'LowerBound', 0, 'UpperBound', 1);
    c_expanded = repmat(c, 1, 1, m);
    obj = sum(sum(sum(c_expanded .* x)));
    \end{lstlisting}
\subsubsection{约束条件}
    等式约束和不等式约束需要分别进行处理
    \subsubsection{等式约束}
    \begin{lstlisting}
    eq_cons = [];
    for i = 2:n+1
        eq_cons = [eq_cons, sum(sum(x(i, 2:n+1, :))) == 1];
    end
    for k = 1:m
        eq_cons = [eq_cons, sum(x(1, 2:n+1, k)) == 1, sum(x(2:n+1, 1, k)) == 1];
    end
    for i = 1:n+1
        for k = 1:m
            eq_cons = [eq_cons, x(i, i, k) == 0];
        end
    end
    \end{lstlisting}
    \subsubsection{不等式约束}
    \begin{lstlisting}
    ineq_cons = [];
    for i = 2:n+1
        for k = 1:m
            ineq_cons = [ineq_cons, sum(d(2:n+1) .* x(i, 2:n+1, k)) <= Q];
        end
    end
    \end{lstlisting}
\subsubsection{解决ILP问题}
    在Matlab中,我们可以直接调用intlinprog
    \begin{lstlisting}
    prob = optimproblem;
    prob.Objective = obj;
    prob.Constraints.Equality = eq_cons;
    prob.Constraints.Inequality = ineq_cons;
    options = optimoptions('intlinprog', 'Display', 'off');
    [sol, fval, exitflag, ~] = solve(prob, 'Options', options);
    \end{lstlisting}
\subsection{结果显示}
    \begin{lstlisting}
    if (exitflag == 0)
        disp('No feasible resolution');
    else
        routes = cell(1, m);
        for k = 1:m
            route = [];
            for i = 2:n+1
                for j = 2:n+1
                    if sol.x(i, j, k) == 1
                        route = [route, i];
                        break;
                    end
                end
            end
            routes{k} = route;
        end
        
        % Total distance
        totalDistance = fval;
    end
    \end{lstlisting}
\subsection{完整代码}
    \begin{lstlisting}
    % Example parameters
    data = load("final_project/model.mat");
    n = data.city;
    m = data.veh;
    c = data.maps;
    n = 5; % Number of customer locations (excluding the depot)
    m = 2; % Number of vehicles
    Q = 10; % Vehicle capacity constraint
    d = [0, 1, 1, 1, 1, 1]; % Demand of each customer point (excluding depot)
    c = [0  17 29 20 21 16;
         17 0  19 15 17 16;
         29 19 0  15 18 16;
         20 15 15 0  22 16;
         21 17 18 22 0  16
         16 16 16 16 16 0]; % Distance matrix (including depot)
        
    % Call the function
    [routes, totalDistance] = solve_CVRP(n, m, Q, d, c);
        
    % Display Results
    fprintf('Optimal total distance: %.2f\n', totalDistance);
    for k = 1:m
        fprintf('Route for vehicle %d: ', k);
        fprintf('%d ', routes{k});
        fprintf('\n');
    end
        
    function [routes, totalDistance] = solve_CVRP(n, m, Q, d, c)
        % SOLVECVRP Solves the Capacitated Vehicle Routing Problem (CVRP)
        %
        % Inputs:
        %   n - Number of customer locations (excluding the depot)
        %   m - Number of vehicles
        %   Q - Vehicle capacity constraint
        %   d - Demand of each customer point (including depot as 0)
        %   c - Distance matrix
        %
        % Outputs:
        %   routes - Cell array containing the routes for each vehicle
        %   totalDistance - Total distance of the optimal solution
            
        % Include depot in the demand array if not already included
        
        % Decision variables: x(i,j,k) indicates if vehicle k travels from i to j
        x = optimvar('x', n+1, n+1, m, 'Type', 'integer', 'LowerBound', 0, 'UpperBound', 1);
        
        % Expand the distance matrix to match the dimensions of x
        c_expanded = repmat(c, 1, 1, m);
            
        % Objective function: minimize the total distance
        obj = sum(sum(sum(c_expanded .* x)));
            
        % Constraints
        eq_cons = [];
        ineq_cons = [];
            
        % Constraint 1: Each customer point must be visited exactly once
        for i = 2:n+1
            eq_cons = [eq_cons, sum(sum(x(i, 2:n+1, :))) == 1];
        end
            
        % Constraint 2: Each vehicle must start and end at the depot
        for k = 1:m
            eq_cons = [eq_cons, sum(x(1, 2:n+1, k)) == 1, sum(x(2:n+1, 1, k)) == 1];
        end
            
        % Constraint 3: Vehicle capacity constraint
        for i = 2:n+1
            for k = 1:m
                ineq_cons = [ineq_cons, sum(d(2:n+1) .* x(i, 2:n+1, k)) <= Q];
            end
        end
        
        % Constraint 4: Ensure x_ii^k = 0 for all i and k
        for i = 1:n+1
            for k = 1:m
                eq_cons = [eq_cons, x(i, i, k) == 0];
            end
        end

        % Solve the problem
        prob = optimproblem;
        prob.Objective = obj;
        prob.Constraints.Equality = eq_cons;
        prob.Constraints.Inequality = ineq_cons;
        options = optimoptions('intlinprog', 'Display', 'off');
        [sol, fval, exitflag, ~] = solve(prob, 'Options', options);
            
        % Extract routes
        if (exitflag == 0)
            disp('No feasible resolution');
        else
            routes = cell(1, m);
            for k = 1:m
                route = [];
                for i = 2:n+1
                    for j = 2:n+1
                        if sol.x(i, j, k) == 1
                            route = [route, i];
                            break;
                        end
                    end
                end
                routes{k} = route;
            end
                
            % Total distance
            totalDistance = fval;
        end
    end
    \end{lstlisting}

